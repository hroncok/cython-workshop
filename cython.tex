\documentclass{beamer}

\mode<presentation> {
  \usetheme{Goettingen}
  \setbeamercovered{transparent}
}

\usepackage[utf8]{inputenc}
\usepackage[english]{babel}
\usepackage{graphicx}
\usepackage{adjustbox}
\usepackage{listings}
\usepackage{courier}
\definecolor{thisblue}{RGB}{51 51 179}
\hypersetup{allcolors=thisblue,colorlinks=true}

\beamertemplatenavigationsymbolsempty

\newcommand*{\pic}[2][]{%
\centering\begin{adjustbox}{max size={\textwidth}{0.85\textheight}}
    \includegraphics[#1]{#2}%
\end{adjustbox}
}

\lstset{language=Python}
\lstset{
 morekeywords={cdef,cpdef,timeit,cython,load_ext,assert,double,NULL,size_t,void,bint,ctypedef,struct,union,short,unsigned,cimport,readonly}
}

\definecolor{mygray}{rgb}{0.4,0.4,0.4}
\definecolor{mygreen}{rgb}{0,0.5,0}
\definecolor{myorange}{rgb}{1.0,0.4,0}

\lstset{
basicstyle=\ttfamily\color{black},
breaklines=true,
commentstyle=\color{mygray},
keywordstyle=\bfseries\color{thisblue},
numberstyle=\color{mygray},
%identifierstyle=\color{mygreen},
showspaces=false,
showstringspaces=false,
stringstyle=\color{myorange}
}

\lstset{literate=%
   *{0}{{{\color{mygreen}0}}}1
    {1}{{{\color{mygreen}1}}}1
    {2}{{{\color{mygreen}2}}}1
    {3}{{{\color{mygreen}3}}}1
    {4}{{{\color{mygreen}4}}}1
    {5}{{{\color{mygreen}5}}}1
    {6}{{{\color{mygreen}6}}}1
    {7}{{{\color{mygreen}7}}}1
    {8}{{{\color{mygreen}8}}}1
    {9}{{{\color{mygreen}9}}}1
}

\title{Cython: Stop writing native Python extensions in C}
\author{Miro Hrončok}
\institute[D2016]{DevConf.cz 2016}


\begin{document}

\begin{frame}
  \titlepage
\end{frame}

%\begin{frame}
%  \frametitle{Contents}
%  \tableofcontents
%\end{frame}


\section{Little exercise}

\begin{frame}
  \frametitle{Little exercise}
    \begin{itemize}
      \item What are your skills?
      \begin{enumerate}
        \item Python beginner
        \item Python almost guru
        \item Python guru not knowing native extensions
        \item Python guru knowing native extensions
      \end{enumerate}
      \item raise your hand with so many fingers
      \item please create pairs (1-4 and 2-3 if possible)
      \item get stuff from \href{https://github.com/hroncok/cython-workshop}{github.com/hroncok/cython-workshop}
      \item get Cython and IPython (for Python 3)
    \end{itemize}
\end{frame}

\section{About Cython}

\begin{frame}
  \frametitle{About Cython}
    \begin{itemize}
      \item \href{http://cython.org/}{cython.org}
      \item programming language
        \begin{itemize}
          \item similar to Python
          \item static typing from C/C++
        \end{itemize}
      \item compiler
        \begin{itemize}
          \item from Cython language to C/C++
          \item to Python extension module
          \item or to standalone apps*
        \end{itemize}
      \item feels like Python
      \item woks like C/C++
    \end{itemize}
\end{frame}

\begin{frame}
  \frametitle{Cython: A Guide for Python Programmers}
    \pic{book.jpg}
\end{frame}

\begin{frame}
  \frametitle{Python extension modules}
    \begin{itemize}
      \item \href{https://docs.python.org/3/extending/extending.html}{Extending Python with C or C++}
      \item I refer to them as ``native extensions'' a lot
      \item performance
      \item interacting with C/C++ libraries/code
      \item \lstinline{\#include <Python.h>}
      \item it hurts to write
      \item it hurts to read
      \item it hurts to maintain
      \item it hurts to keep it compatible with both Pythons
      \item Cython makes all this easy
    \end{itemize}
\end{frame}

\section{Fibonacci}

\begin{frame}[fragile]
  \frametitle{Fibonacci}
    \begin{lstlisting}
def pyfib(n):
    a, b = 0, 1
    for i in range(n):
        a, b = a + b, a
    return a


# In IPython, try to run it like:
%timeit pyfib(190)
    \end{lstlisting}
\end{frame}

\begin{frame}[fragile]
  \frametitle{Fibonacci}
    \begin{lstlisting}
def pyfib(n):
    a, b = 0, 1
    for i in range(n):
        a, b = a + b, a
    return a


def cyfib(int n):
    cdef int i
    cdef long a = 0, b = 1
    for i in range(n):
        a, b = a + b, a
    return a
    \end{lstlisting}
\end{frame}

\begin{frame}[fragile]
  \frametitle{Fibonacci}
    \begin{lstlisting}
%load_ext cythonmagic

%%cython
def cyfib(int n):
    cdef int i
    cdef long a = 0, b = 1
    for i in range(n):
        a, b = a + b, a
    return a

%timeit cyfib(190)
    \end{lstlisting}
\end{frame}

\section{setuptools}

\begin{frame}[fragile]
  \frametitle{setuptools}
    \begin{lstlisting}
from setuptools import setup, Extension
from Cython.Distutils import build_ext

setup(
    cmdclass={'build_ext': build_ext},
    ext_modules=[
        Extension('fib', ['fib.pyx'])],
    classifiers=[
        'Programming Language :: Cython'],
)
    \end{lstlisting}
\end{frame}

\section{Types}

\begin{frame}[fragile]
  \frametitle{Typing}
    \begin{lstlisting}
a = [x + 1 for x in range(12)]
b = a
a[3] = 42.0
assert b[3] == 42.0
a = 13
assert isinstance(b, list)
    \end{lstlisting}
\end{frame}

\begin{frame}[fragile]
  \frametitle{Typing}
    \begin{lstlisting}
def dummy_func():
    cdef int i
    cdef int j
    cdef float k

    j = 0
    i = j
    k = 12.0
    j = 2 * k
    assert i != j
    \end{lstlisting}
\end{frame}

\begin{frame}[fragile]
  \frametitle{Typing}
    \begin{lstlisting}
def several_at_once():
    cdef int i, j, k
    cdef float price, margin
    #...

def optional_initial_value():
    cdef int i = 0
    cdef long int j = 0, k = 0
    cdef float price = 0.0, margin = 1.0
    #...
    \end{lstlisting}
\end{frame}

\begin{frame}[fragile]
  \frametitle{Pointers, Arrays, bint...}
    \begin{lstlisting}
cdef int *p
cdef void **buf = NULL
cdef void (*func)(int, double)

cdef size_t arr[3]

cdef double golden_ratio
cdef double *p_double

p_double = &golden_ratio
p_double[0] = 1.618
print(golden_ratio)
print(p_double[0])

cdef bint ok
    \end{lstlisting}
\end{frame}

\begin{frame}[fragile]
  \frametitle{Structs, union}
    \begin{lstlisting}
cdef struct coord:
    float x
    float y
    float z

cdef union uu:
    int a
    short b, c

ctypedef struct coord:
    #...

ctypedef union uu:
    #...

cdef uu myvar
    \end{lstlisting}
\end{frame}

\begin{frame}[fragile]
  \frametitle{Structs initialization}
    \begin{lstlisting}
cdef coord a = coord(0.0, 2.0, 1.5)

cdef coord b = coord(x=0.0, y=2.0, z=1.5)

cdef coord c
c.x = 42.0
c.y = 2.0
c.z = 4.0

cdef coord d = {'x':2.0,
                'y':0.0,
                'z':-0.75}
    \end{lstlisting}
\end{frame}


\section{Functions}

\begin{frame}[fragile]
  \frametitle{Cython's Three Kinds of Functions}
    \begin{itemize}
      \item \lstinline{def}
      \item \lstinline{cdef}
      \item \lstinline{cpdef}
    \end{itemize}
    \begin{lstlisting}
def cyfib(int n):
    cdef int i
    cdef long a = 0, b = 1
    for i in range(n):
        a, b = a + b, a
    return a

cdef long cyfib(int n):
    #...

cpdef long cyfib(int n):
    #...
    \end{lstlisting}
\end{frame}



\section{Classes}

\begin{frame}[fragile]
  \frametitle{Classes and Extension Types}
    \begin{lstlisting}
from libc.stdlib cimport malloc, free

cdef class Matrix:
    cdef readonly unsigned int nr, nc
    cdef double *_matrix

    def __cinit__(self, nr, nc):
        self.nr, self.nc = nr, nc
        self._matrix = <double*>malloc(nr * nc * sizeof(double))
        if self._matrix == NULL:
            raise MemoryError()

    def __dealloc__(self):
        if self._matrix != NULL:
            free(self._matrix)

    \end{lstlisting}
\end{frame}


\section{Knapsack}

\begin{frame}
  \frametitle{Knapsack Problem}
    \begin{itemize}
      \item problem from combinatorial optimization
      \item \href{https://en.wikipedia.org/wiki/Knapsack_problem}{Wikipedia}
      \item you have a knapsack with a given capacity
      \item you get bunch of items with costs and weights
      \item get the most expensive combination that fits in
      \item bruteforce is $O(2^n)$
      \item get it from \href{https://github.com/hroncok/cython-workshop}{github.com/hroncok/cython-workshop}
    \end{itemize}
\end{frame}

\begin{frame}[fragile]
  \frametitle{Knapsack Problem}
    \begin{lstlisting}
sack = {
    'id': 1,
    'count': 15,
    'capacity': 18.5,
    'items': [
        {'weight': 2.0, 'cost': 2.5},
        #...
    ],
}

solver = BruteSolver(sack)
maxcombo, maxcost = solver.solve()
    \end{lstlisting}
\end{frame}


\section{Further information}

\begin{frame}
  \frametitle{Further information}
    \begin{itemize}
      \item \href{http://docs.cython.org/}{docs.cython.org}:
      \begin{itemize}
        \item \href{http://docs.cython.org/src/tutorial/clibraries.html}{Using C libraries}
        \item \href{http://docs.cython.org/src/tutorial/numpy.html}{Working with NumPy}
        \item plenty of other topics
      \end{itemize}
      \item \href{http://shop.oreilly.com/product/0636920033431.do}{Cython book}
      \item \href{https://github.com/cythonbook/examples}{examples from the book}
      \item contact me
      \begin{itemize}
        \item miro@redhat.com
        \item mhroncok @ freenode
        \item @hroncok on Twitter or GitHub
      \end{itemize}
    \end{itemize}
\end{frame}


\end{document}

